%\documentclass[12pt]{article}
%\usepackage[margin=1 in, head=0.9 in]{geometry}
%\usepackage{fancyhdr}
%\usepackage{listings}
%\usepackage{caption}
%\usepackage{color}
%\usepackage{xcolor}
%\usepackage{caption, apacite}
%\DeclareCaptionFont{white}{\color{white}}
%\DeclareCaptionFormat{listing}{\colorbox{gray}{\parbox{\textwidth}{#1#2#3}}}
%\captionsetup[lstlisting]{format=listing,labelfont=white,textfont=white}
%\usepackage{graphicx}
%\usepackage{amsmath, amssymb, amsthm}
%\usepackage[all,cmtip]{xy}
%\pagestyle{fancy}
\input{/home/dmitry/Work/Research/thesis/FINALE/settings.tex}

\begin{document}

\title{Diagnosing angular (directional) spectra by Tikhonov regularization (ridge regression, tapered least squares) in numerical studies of wave scattering (wave-topography interaction)}
\maketitle

\section*{Abstract}
Realistic numerical simulations of wave propagation (both surface and internal) involve complex patterns of interference that conceals energy transfers. The elementary wave components can be deduced from angular spectrum. Here, the energy distribution is found by solving an inverse model based on plane wave dynamics. The solution method is a least square minimization augmented with penalizing term (Tikhonov regularization). This aids stability in presence of noise and provides better spectral resolution for under-sampled signal. Further, applicability of the method for energy estimates is tested against analytical solution of wave scattering by bottom topography. In the first problem a periodic shallow gravity wave attacks a circular island. The inverse model correctly identifies energy lobes and their amplitude, hence, allowing to examine scattering amplitude in more complex environments. In the second test, a low mode internal wave reflects from a continental slope represented as step topography. Estimated bulk reflection coefficient compares excellent with the analytical solution. But in case of three dimensional topography the method as well provides information on spatial variation of point-wise coefficient due to inhomogeneity of the prescribed incident field. The results lend confidence in robustness of the inverse technique, hence, suggesting its applicability to field observations of the real ocean.

\section{Introduction}
The oceanic waves interfere and create obscure spatial characteristic of currents and pressure. The 
staggered pattern makes difficult to answer the simplest questions about 
waves: from where are they coming and how strong they are. Hence, there will always be an 
uncertainty in estimate of energy transfers and budgets. In an example of a standing wave, no 
energy transfer is taken place, yet two waves are present but oppositely traveling. Another example 
concerns spatially (horizontally) tight beams of energy transport emerging from as wave-topography 
interaction. Such phenomena was observed in numerical simulations of tsunami waves 
(\cite{tang2012direct}) and internal waves of tidal frequency (internal tides) 
(\cite{simmons2004internal}, \cite{arbic2010concurrent}(?)) (and in observations 
(\cite{zhao2016global})). Both cases show inevitable difficulties in studies of wave-topography 
interaction. Energy is transfered into scattered wave field that undoubtedly has large number of 
components. This might produce  To present a method to deduce dynamics of a complex wave 
field such as aforementioned ones is a chief reason for this note (chapter, paper).\\% And it is 
%application to problems of tsunami wave scattering by submarine island and reflection of low-mode 
%internal tide.\\

A straightforward method is to obtain spatial frequencies of elementary waves (\cite{barber1963directional}) by means of two dimensional Fourier transform. It further can be mapped onto a circle leading to angular spectra. In the field studies this approach was extensively used in relation to surface wave phenomena, e.g. propagation (\cite{munk1963directional}, reflection (\cite{dickson1995wave}, \cite{thomson2005reflection}, island diffraction (\cite{pawka1983island}). The spectral description was also applied to tsunami trapping (\cite{romano2013wavenumber}) and internal tide propagation (\cite{hendry1977observations}, \cite{lozovatsky2003spatial}). Estimation of spectra is a problem that one encounters and here is dealt by inverse modeling (\cite{long1986inverse}). As discrete Fourier transform leads to ambiguity in elementary wave direction, additional inferences are necessary. Even though application of the Hilbert transform can resolve the issue (\cite{mercier2008reflection}) \footnote{(Will it work with many components? In presence of corrugated surfaces?)}, it does not utilize all wave field characteristics available. Logically, constrains of wave dynamics are expected to lessen Gibbs phenomena, increase directional resolution and provide a way to super-resolution (\cite{kay1981spectrum}, \cite{sacchi1998interpolation}) so essential (viable, relevant, crucial) for resolving short-crested waves.\\
%the constrains of wave dynamics are expected to lessen

The inverse approach has to comprise dynamical statements of the wave field. So that angular 
spectrum will be related to (translated into) observed pressure and currents simultaneously. But 
such postulated problem leads to overdetermined and ill-posed system of equations, naive least 
square solution is unstable in relation to small changes or errors in data. For example, errors in 
numerical simulations might arise due to numerical and physical friction (reference?) leading to 
nonuniform wave amplitudes over sampling window of the numerical ocean. Hence, regularization is 
generally necessary to stabilize and find a physically meaningful estimate (\cite{munk2009ocean}, 
\cite{snieder1999inverse}). In studies of wind wave spectra this is a well 
known approach when penalizing terms are added to cost functions. These terms can comprise 
constrains on spectrum shape and smoothness (\cite{long1979variational}, 
\cite{herbers1990estimation}), spectrum sparseness (\cite{hashimoto1989directional}, 
\cite{sacchi1996estimation}\footnote{this is geophysics reference}, or statistical properties 
(\cite{benoit1997analysing}). Unfortunately, the mentioned methods cannot be utilized in studies of 
wave-topography interaction. The scattered field is ``phase-locked" with incident one and thence, 
phase must be retained in the model equations (e.g. \cite{thomson2005reflection}) leading to 
deterministic wave decomposition. This is in contrary to wind waves of open sea that are mainly 
stochastic.\\

The simplest regularization technique, nevertheless powerful one seeks an inverse estimate of minimum variance. Initially developed by \cite{tikhonov2013numerical} for solution of Fredholm integral equations, Tikhonov regularization has a wide application starting from theoretical work in acoustics (\cite{colton1996simple}) and holography (\cite{williams2001regularization}) to observational oceanography (\cite{munk2009ocean}) and geophysics (\cite{snieder1999inverse}) where it is named ''tapered" or ``damped" least squares. But it has never been used \footnote{Dushaw used similar inverse} to oceanic wave problems. In the first section the inverse model will be formulated in order to find angular spectrum of waves. The developed approach than (Section 3) is illustrated on simple synthetic experiments to show its performance characteristics. And in later sections (4 and 5) is applied to numerical experiments of wave scattering where comparison with the well-known analytical solution is taken place. In section 6 concluding remarks are made and drawbacks are outlined.\\
\footnote{Should I add plane-wave fit? ``subtraction" method of Jody (and Zygmunt)?}
~\\

\section{Regularization technique of obtaining angular spectrum}
\subsection{Mathematical preliminary}
Let pressure or sea level in complicated seas to be described by angular spectrum\footnote{This is 
not spectrum per se, is term usage still possible? Longuet-Higgins could be cited here},
\begin{equation}
\label{C1:eq.spectrum}
p(\vec{r}, t) = \int_0^{2\pi}  d\theta_k S(\theta_k) e^{i \vec{k}(\theta_k) \cdot \vec{r} + \phi(\theta_k) - i \omega t}
\end{equation}
Here each elementary (monochromatic) sine wave of wavenumber $k$ travels in direction $\theta$ with 
energy $S(\theta)^2 d\theta$ and temporal (spatial) lag of $\phi(\theta)$. Through out the note 
temporal dependence is omitted and complex notation is inferred. It is sought to estimate the 
angular spectrum $S(\theta)$ and phase distribution $\phi(\theta)$ from a finite set of 
observations. At first, this problem is reformulated in terms of Fourier coefficients 
(\cite{munk1963directional}, \cite{benoit1997analysing}, \cite{rafaely2004plane}) by application of 
Jacobi-Anger expansion,
\begin{equation}
p(r, \theta) = e^{i \vec{k}(\theta) \cdot \vec{r}} = \sum_{m = -\infty}^{m = \infty} i^{m} J_{m}(k r) e^{im(\theta - \theta_k)}
\end{equation}
shows that a field at point $(r, \theta)$ produced by plane wave can be expanded in series of Bessel functions and circular functions. Than its substitution into \eqref{C1:eq.spectrum} and reorganization lead to
\begin{equation}
\label{C1:eq.series}
p(r, \theta) = \sum_{m=-\infty}^{m=\infty} \big[ \int_0^{2\pi}  d\theta_k S(\theta_k) e^{i\phi(\theta_k)} e^{-im\theta_k} \big] i^m J_m(kr) e^{im\theta}
\end{equation}
Term in brackets (square brackets) represent convolution integrals defining Fourier coefficients of 
order $m,~A_m - i B_m$. Thence, series \eqref{C1:eq.series} state a model equation to find the 
unknown coefficients from the known, measured pressure field. Its spatial distribution can be 
sampled at a set of points $(r_i, \theta_i)$ and if infinite series is truncated at order $N$, 
than matrix equation is found,
\begin{equation}
\label{C1:p.eq}
p_i = \sum_{m = -N}^{m = N} J_m(k r_i) e^{im(\theta + \pi/2)} (A_m  - i B_m)
\end{equation}
Real and imaginary parts constitute two separate problems allowing deterministic definition of the spectrum. Angular spectrum and phase distribution will be recomposed as usual by
\begin{align*}
S(\theta_k) e^{i \phi(\theta_k)} = \frac{1}{\pi} A_0 + \frac{2}{\pi} \big[ \sum_{m = 1}^{m = N} A_m \cos m\theta_k + i B_m \sin m\theta_k \big]
\end{align*}
The same steps are repeated but with current velocities instead. By invoking plane wave dynamics 
transfer functions are inserted into \eqref{C1:eq.spectrum},
\begin{align}
\begin{Bmatrix}
u(\vec{r}, t) \\ v(\vec{r}, t)
\end{Bmatrix}
=\int_0^{2\pi} d \theta_k \frac{k}{\rho_0 (\omega^2 - f^2)} 
\begin{Bmatrix}
\omega k \cos \theta_k + i f \sin \theta_k \\ \omega k \sin \theta_k - i f \cos \theta_k
\end{Bmatrix}
S(\theta_k) e^{i \vec{k}(\theta_k) \cdot \vec{r} + \phi(\theta_k)}
\end{align}
\textbf{Than} dependence of currents on wave bearing causes splitting of Fourier coefficients and 
asymmetry via Coriolis effect,
\begin{align}
\label{C1:uv.eq}
\begin{Bmatrix}
u_i \\ v_i
\end{Bmatrix}
= \frac{1}{2} \sum_{m = -N}^{m = N} J_{m} (kr_i) e^{im(\theta + \pi/2)}
\begin{Bmatrix}
(\omega - f) A_{m + 1} + (\omega + f) A_{m - 1} - i [(\omega - f) B_{m + 1} + (\omega + f) B_{m - 1}] \\ 
(\omega - f) B_{m + 1} - (\omega + f) B_{m - 1} + i [ (\omega - f) A_{m + 1} - (\omega + f) A_{m - 1}]
\end{Bmatrix}
\end{align}
This results points out that to describe velocity field higher circular harmonics have to be employed. Physically, velocity field has higher spatial variability. This is well known idea since spatial differentiation acts as a high-wavenumber filter (e.g. \cite{rhines1977dynamics}). But in \eqref{C1:uv.eq} additionally, the asymmetry is observed for clockwise and counterclockwise components.\footnote{This is bad, but discussion is necessary}

\subsection{Inverse technique formulation}
Inverse model sets its goal an estimation of angular spectrum from simulated \footnote{observed} 
wave field characteristics. This relation is posed by \eqref{C1:p.eq} and \eqref{C1:uv.eq} and 
formally expressed as,
\begin{equation}
y = K x
\end{equation}
where $y$ measured pressure and currents, $x = {A_m, B_m}$ - Fourier coefficients and $K$ - 
physical model coefficients. The statement does not need to be strictly satisfied as the model 
might not be fully correct such as in case of non-planar waves and presence of errors. 
Additionally, $K$ does not have to be a square matrix, it is possible to have less measurements 
than unknowns. To solve the ill-posed problem, measure of solution's success should be imposed. For 
example, in case of least-square minimization, cost function $J = \{ ||K x - y||^2_2 \}$ is 
minimized so the solution is closest to data points. And the least-squares solution  
will be found as Moore-Penrose inverse (\cite{bennett1992inverse}). Generally, it is unstable to 
small errors in data and produce physically inconsistent results. To avoid the issues in Tikhonov 
regularization a solution should be a minimizer for,
\begin{equation}
\label{C1:Tikh_prob}
J = ||K x - y||^2_2 + \alpha ||x||^2_2
\end{equation}
The second term controls amount of solution's variance with a weight $\alpha$ called a regularization parameter. It can be shown that this parameter represents a low-pass filter over orthonormal vectors of $K^TK$ (\cite{williams2001regularization}). This becomes clear as solution is explicitly expressed,
\begin{equation}
\label{C1:Tikh_solution}
x_{\alpha} = (K^T K + \alpha I)^{-1} K^T y
\end{equation}
The factor $\frac{\lambda_k^2}{\lambda_k^2 + \alpha}$ ($\lambda_k$ - singular value of $K$) 
is introduced to damp small singular values that cause instability of Moore-Penrose solution. 
At the same time, this will exert dependence of the inverse on the regularization parameter: for 
$\alpha = 0$ regular least-square estimate is obtained, while for large $\alpha$ minimizer tends to 
zero. Its choice cannot be made a priori and has to be data-driven. In field studies this is 
usually set by a signal-to-noise ratio (\cite{munk2009ocean}), since the parameter scales noise 
variance (residue) to actual signal's strength. To define $\alpha$ more elaborate techniques exist 
(\cite{williams2001regularization}), but no silver bullet exists. Here straightforward 
approach is adapted that based on trade-off curve (\cite{hansen1993use}).\\
In Tikhonov regularization amount of allowed error is competing with solution's variance. An 
optimal parameter should balance these factors. This is seen as a rapid change in behavior of curve 
associating residue with model's norm as regularization varies (\mfref{C1:F1.lcurve}). In most 
cases the curve has a sharp corner connecting aforementioned limits, hence, the method's name is a 
L-curve (\cite{hansen1999curve}). And its corner is said to occur for an optimal regularization 
parameter.\\

\subsection{Synthetic example to illustrate method}
The so-described method is not different as a regular Discrete Fourier transform of finite time 
series. Here, the spectrum is found over data-points organized in circular antenna, only spatially 
finite region is known (Figure 2a). The solution will have a finite number of orders and this will 
result in Gibbs ringing (Figure 2c). The role of regularization is seen in oversmooothing of the 
spectra and decrease of spectral resolution. In case of perfect model penalized least squares will 
always worse than direct fit. But as some inconsistency appear in the data (modeled here as white 
noise with 1\% variance of main signal) the least squares result in unphysical spectrum. On the 
other hand, white noise improves the penalized least squares solution.\\
It is apparent that maximum order is a crucial characteristic for resolution. From the expression 
(ref-to-spectrum reconstruction) it is seen that two wave will be separated in principle if $ 
\delta \theta \leq \frac{\pi}{N_{max}} $ (\cite{rafaely2004plane}). In the inverse model 
formulation this directly depends to Bessel function. And since these coefficients depend order and 
$ k d_{antenna} $ only later can increase resolving power of the method. But if velocities are used 
in conjunction with pressure observations the order can be increased at least by one as follows 
from the inverse model formulation (\ref{C1:uv.eq}) where there is a shift between Bessel function 
and coefficients. In the following example two waves cannot be discerned since antenna size is two 
small to satisfy Rayleigh criteria, $ sin \delta theta = 1.22 \frac{\lambda}{d_{antenna}} $, so the 
reconstructed spectrum shows one smeared wave. But if velocities are augmented to the data, there 
is an increase in resolving power and waves are actually resolved.\\
Note in the numerical application observed velocities are scaled so that variance of 
residue will have similar magnitude with pressure. \\

\
Figure 2a: Antenna of data-points sampling a synthetic plane wave directed at $180^{\circ}$.\\
Figure 2b: Solution coefficients for the synthetic example: regular least squares, PLS, and with 
white noise added.\\
Figure 2c: Reconstructed spectrum of the wave field.\\

Figure 3a: Two wave interference with observational antenna not able to separate waves.\\
Figure 3b: Solution coefficients for inverse solution without and with velocity components used.\\
Figure 3c: Reconstructed spectrum of the wave field.\\

\section{Test problems of wave scattering}
The aim of this paper is to obtain spectrum which later is used to infer physical properties of 
wave-topography interaction. In the case of wave interaction there is an energy transfer from 
incident wave into scattered. And spectrum can be used to delineate this energy process. Usually, a 
researcher is interested in spatial characteristics of scattered wave field (cite Sato?) or in some 
bulk quantity such as reflection coefficient (\cite{klymak2016reflection}). These two problems will 
be studied by application of the proposed method. The simplest problem are taken for which 
analytical solution is well known, so that comparison between the method result and exact solution 
could be made. Additionally, auxiliary experiment is carried out with no topography which result is 
subtracted from the total field so that exact scattered wave in terms of numerical experiments can 
be studied.

\subsection{Example diffraction of shallow-gravity wave}
The infitinite periodic wave train is sent impinging on a seamount represented by a cylinder of 
radius 40 km rising from plane bottom ocean of depth 5km to the depth of 250 m. The 3 experiments 
of different period are performed. In the numerical experiment it is used the simplified numerical 
code developed by (\cite{kowalik2005numerical}). All nonlinear terms were neglected, while bottom 
friction is controlled by constant coefficient of $3.3 \cdot 10^{-3}$. The integration is carried 
out over grid with resolution of $2~km$ by $2~km$. The boundaries are open, so all scattered waves 
are traveled away without reflection. The periods are taken to be 10, 15 and 20 min so 
that different regimes in response can be caught such as Diffraction and No-interaction.\\
After quasi-equilibrium is reached temporal Fast Fourier transform is employed to find amplitude 
and phase of the strongest wave at each grid point (Figure 4a). Than Fourier coefficients are fed 
into inverse model so that directional spectrum is estimated. The antenna is organized by 
concentric circles of radius $ \lambda/2,~\lambda/4,~\lambda/8 $ where $ \lambda $ is a wavelength 
found from dispersion law of shallow-gravity waves. Each circle is discretized in 100 points that 
are projected to the nearest grid point, so actual number of points per circle might be smaller.\\
The total field is a clear pattern of interference fringes following circularly spreading scattered 
from the seamount waves. At the same time due to large wavefront compare to wavelength there are 
diffractive maxima in the incident wavefield. It is clear that the wavefield is complex. It is 
unclear how much energy was transmitted into the scattered components and what its spatial 
structure. Here the scattered wavefield is reconstructed from spectrum as averaged sealevel emitted 
from the seamount in direction where spctrum is found (Figure 4a).\\

Figure 4a: Amplitude of Fourier coefficient of the total field of sea level. The antenna used at 
each grid point is shown as well. Period of the forced wave is 10 min.\\
Figure 4b: The scattered wave as Filtered out by application of the inverse method.\\
Figure 4c: The scattered wave as found by subtraction method.\\

Figure 5: Comparison of normalized scattering amplitude found by the inverse method and from 
subtraction method with analytical solution.

\subsection{Results and discussion}

\subsection{Example: reflection of internal tide from step discontinuity}
In the next problem an internal tide reflection from simplified topography is studied. The internal 
tide is given by the first baroclinic mode prescribed on the right side of channel with.\\
The MITgcm is employed where a nonrotating channel ends with discontinuity whose depth is changed. 
The channel 
depth is constant of 2000 m and ocean is uniformly stratified with Brunt-Vaisala frequency $ N^2 = 
1.5 \cdot 10^{-5} $. The horizontal resolution is 8 km. In vertical there are 25 equally 
discretized levels. Around the box sponges of 6 grid cells are set to absorb reflected waves.\\
On the right side a low-mode internal tide is prescribed with frequency of $ M_2 $ harmonic as 
perturbation in temperature profile and velocity field following respective mode-1 eigenfunction 
structure. Horizontally the basic field is perturbed over 450 km section to decrease diffractive 
effects. Additionally, the so-generated beam is let to freely propagate to get rid of edge effects 
and obtain spatially uniform field. The beam than impinges on a step topography where reflection  
and scattering into higher modes take place (Figure 6).\\
The model calculated fields are saved over two periods, harmonically analyzed. From temperature 
field one can obtain baroclinic pressure anomaly (\cite{Kelly2010}). This then are fit to eigenmode 
structure functions. So per each dynamical field quantity a complex amplitude is found. These are 
further used in spectrum estimate. The antenna for analysis is built in the same way as in previous 
example.\\
In such problems one is usually interested in estimate of energy being reflected. The reflected 
wave is defined here as all easterly propagating waves, i.e. compass directions located in the 
first and fourth quadrant. The energy flux is than can be reconstructed for these directions and 
compared with subtraction method (Figure 6b).\\
Energy conservation\\
The bulk reflection coefficient on the other hand can be obtained as a ratio of total incident 
energy going through a cross section and total reflected energy traveling away. As additional check 
this can be compared with results of analytical solution (\cite{chapman1981scattering}) (Figure 
7).\\

\subsubsection{Test description}
\subsubsection{Results and discussion}

%%%% FIGURES %%%%
\newpage
\section{Figures}
\begin{figure}
	\centering
	\mfig[0.75]{L_curve_example.png}
	\caption{L-curve example}
	\label{C1:F1.lcurve}
\end{figure}

\begin{figure}
	\mfig[0.75]{uni_field.png}
	\mfig[0.75]{solution_comparison.png}
	\caption{(a) Synthetic sea level field with added white noise. Yellow dots show data points 
	used in inversion. (b) Fourier coefficients found as by regular least squares and penalized 
	least-squares without and with added noise. (c) Reconstructed spectrum for the considered 
	methods.}
	\label{C1:F2.1w}
\end{figure}

\begin{figure}
	\label{C1:F2.2w}
	\centering
	\mfig[0.75]{uvp_field.png}
	\mfig[0.75]{solution_comparison_2w.png}
	\caption{(a) Synthetic sea level field of two interfering waves with added white noise. Yellow 
	dots show data points used in inversion. (b) Fourier coefficients found inversion with 
	penalized least-squares method when only sea level data are used and with currents. (c) 
	Reconstructed spectrum for the considered problems.}
\end{figure}

\begin{figure}
	\centering
	\mfig[1]{example_field.png}
	\caption{(a) Result of numerical experiment. (b) Scattered field alone with superposed 
		analytical solution}
\end{figure}

\begin{figure}
	\centering
	\mfig[0.25]{averaging_spectrum.png}
	\caption{To definition of scattering amplitude from directional spectrum.}
\end{figure}

\begin{figure}
	\centering
	\mfig[1]{spectrum_scattered.png}
	\caption{(a) Scattered field obtained from partial integration of spectrum. (b)-(d) Comparison 
		of averaged scattering amplitude found by spectrum approach with exact scattered field and 
		analytical solution in different regimes.}
\end{figure}

\begin{figure}
	\centering
	\mfig[1]{robustness_spectrum.png}
	\caption{(a) Stability of scattering amplitude to change in averaging sector (row - (a)) and 
		fitting window (row - (b)) for different wave periods. The green line shows exact scattered 
		field.}
\end{figure}

\begin{figure}
	\centering
	\mfig[0.75]{oblique_reflection.png}
	\caption{Oblique incidence experiment. (a) Distribution of incident field (a) and reflected 
		(b). Insets show the corresponding spatial fields for depth ratio of 0.5.}
\end{figure}

\begin{figure}
	\centering
	\mfig[0.75]{robustness_reflection.png}
	\caption{(a) and (b) Comparison of reflection coefficient between analytical theory, exact 
		numerical solution and spectrum obtained estimate. Here results for different fitting 
		window 
		are given. (c) and (d) error in energy budget for incident (circles) and reflected 
		(squares) 
		and for different antenna sizes. Left column is for straight incidence experiments and 
		right 
		column for oblique incidence.}
\end{figure}
\bibliographystyle{apacite}
\bibliography{/home/dmitry/Bibtex_lib/my_first_lib}

\end{document}

%The simplest way to identify the scattered components is to numerically remove bottom disturbance 
%(\cite{kowalik2008kuril}, \cite{klymak2016reflection}). Than set new experiment to obtain 
%differences which will provide insights into distribution of energy. This method can provide 
%satisfactory results only if ``subtraction" of bathymetry is possible, that is there is no 
%surrounding complexity. And if one is not interested in the energy transfers occurring over the 
%topography which can be a case for a study on trapped wave motions. As well in complex numerical 
%simulations when mesoscale conditions is simulated in the same time as wave phenomena, clearly 
%removement of bathymetry will lead to different background physical conditions such that 
%interaction between waves and background conditions will be erroneous.\\
%In classical observational wave () literature premises of linear wave field are used to obtain 
%directional spectra, $|A(\sigma, \vec{k})|$. Fourier transform can be carried out in order to 
%obtain frequency-wavenumber spectra. This defines distribution of wave energy in 
%frequency-wavenumber space differential intervals. To obtain this different methods might be 
%employed (). Here the trivial approach is considered based on plane-wave fit methods (). The basic 
%approach to find spectral representation is to use underlying wave dynamics and use plane wave 
%fit. 
%This is a traditional approach in analysis of satellite altimetry observations 
%(\cite{ray2001estimates}, \cite{zhao2016global}). In application of regular plane wave fit though 
%there is a drawback that only finite number of components can be resolved. This problem arises as 
%least squares method can infinitely increase amplitude in order to decrease misfit. This will lead 
%to unphysical results. One of the ideas to solve this problem is setting additional restriction on 
%amplitude. So this is why obtaining non-negative wavenumber spectra will bring better 
%performance.\\
%In order to show feasibility of directional wave spectra in numerical experiments we proceed as 
%follows. In section it is given the basic definition and description of numerical procedure. As 
%well it is studied its characteristics on a synthetic data. With outlined procedure comparison of 
%wavenumber spectra to an analytical solution of SGW diffraction by seamount is carried out 
%(Section 
%3a) and comparison with reflection of internal tide from step-like continental slope is done with 
%comparison with an appropriate analytical solutions (Section 3b). In section 4 concluding remarks 
%are made.\\
%~\\

%Bessel functions decay... super fast\\
%The so-chosen basis functions are not orthogonal, this leads to ill-conditioned matrix K, the 
%solution is unstable for small changes in the data. Some sort of regularization is necessary. Here 
%we choose the simplest of most is Tikhonov regularization. The minimization problem to solve is 
%(or 
%to use cost function)
%
%Here the first term is part of regular least-squares solution and represents minimization of 
%residual. The second term minimizes variance. One tries to find a balanced solution that has 
%minimum residual and smallest norm trade-off parameter.
%But success and resultant fit depends on a regularization parameter. The parameter acts as a high 
%pass filter over eigenvectors of $K$ (\cite{williams2001regularization})\\
%In case of rapidly varying bathymetry a local cutoff to drop stations is based on WKB-alike 
%condition, i.e. variations in wavenumber are much smaller compare to its value.\\
%The formed matrix equations is solved by means of linear programming methods 
%\cite{haskell1981algorithm}.\\


%\section{Application}
%\subsection{Tsunami wave diffraction by a round island}
%Now the method is tested against known analytical problems. The surface gravity wave interaction 
%with submarine seamount. THe solution is known for many years (\cite{longuet1967trapping}).\\
%In the numerical experiment it is used the simplified numerical code developed by 
%(\cite{kowalik2005numerical}). All nonlinear terms were neglected, while bottom friction is 
%controlled by constant coefficient of $3.3 \cdot 10^{-3}$. The integration is carried out over 
%grid 
%with resolution of $2~km$ by $2~km$. The continuous harmonic wave train is sent towards the shoal 
%of depth $500~m$ in sea of depth $5300~m$ from the east side. After the numerical experiment 
%achieved equilibrium the sea level and horizontal currents are sampled. Fourier transform is done 
%and than the presented method here is used to build directional distribution in the wave field. 
%Example map of energy flux is presented on Figure 3.\\
%\begin{figure}
%\mfig[0.5]{fluxes_circ_sm.png}
%\caption{The simulated energy flux maps for SGW diffraction over the shoal.}
%\end{figure}
%The directional decomposition shows 
%\begin{figure}
%\mfig[0.5]{spectra_windrose_circ_sm.png}
%\caption{The simulated energy flux maps for SGW diffraction over the shoal.}
%\end{figure}
%The analytical solution is shown on Figure 4.
%\begin{figure}
%\mfig[0.5]{scatt_LH_periods.png}
%\caption{analytical solution}
%\end{figure}
%
%\subsection{Internal tide reflection from step}
%For internal tide reflection problem it is taken MITgcm numerical model initialized with constant 
%stratification. The continental slope is represented as depth discontinuity arising from flat 
%bottom at 2000 m to variable depth. The numerical results after the equailibrium was achieved are 
%sample, harmonically fitted to $M_2$ frequency. Than eigenmode decomposition is done. Example of 
%the obtained results is shown by Figure ...\\
%\begin{figure}
%\mfig[0.5]{just_interf.png}
%\caption{Interference due to oblique reflection.}
%\end{figure}
%The solution is compared against \cite{chapman1981scattering} and is shown by
%\begin{figure}
%\mfig[0.5]{refl_coef_smp.png}
%\caption{Reflection coefficient}
%\end{figure}
%
%\section{Discussion and Conclusions}
%\begin{itemize}
%\item Energy preservation
%\item No phase reconstruction
%\item Discrete since no a priori model is used
%\item Useful tool for numerical experiment analysis
%\end{itemize}
%
%
%\section{Problems}
%\begin{itemize}
%\item I am confusing spectra and amplitude, as well cross-spectrum and covariance
%\end{itemize}
%\section{TO DO LIST}
%\begin{itemize}
%\item Right now my questions are weak. Primarily, it comes from my explanation, why do you need 
%something new? My major answer: a) regular plane wave fit fails for multiple wave components; b) 
%in 
%complex models - it is not that easy to subscribe bathymetry. \textbf{That should be my 
%arguments!} 
%Explore them in more detail. Ref: Zhao method, Mercier, Jody's subtraction.
%
%\item Monte Carlo and estimate of errors - first thing to do
%
%\item Solution for wave scattering by an elliptic seamount
%
%\item What is physical meaning of Sf? Ref: Wagner, IT scattering
%
%\item Set Chapman experiments
%
%\item Set tsunami experiments with Koko Guoyt.
%\end{itemize}
%\section{pieces}
%Here the cross-spectral components of observations at points $i$ and $j$; $S(\theta)$ - 
%directional wave spectra in direction $\theta$ and . Here for specifying this relation the regular 
%plane wave representation with corresponding polarization relations is adopted. For example, 
%cross-spectra between velocity at point $i$ and pressure at point $j$ (somewhat related to "energy 
%flux") will be expressed as\\
%
%with $\vec{k}(\theta) = k (\cos(\theta), \sin (\theta))$ and $\vec{x}_i,~\vec{x}_j$ will be radius 
%vectors to points $i$ and $j$. Numerically, the integral representation is disretized and linear 
%model to solve at each computation grid cell\\

