%\documentclass[12pt]{article}
%\usepackage[margin=1 in, head=0.9 in]{geometry}
%\usepackage{fancyhdr}
%\usepackage{listings}
%\usepackage{caption}
%\usepackage{color}
%\usepackage{xcolor}
%\usepackage{caption, apacite}
%\DeclareCaptionFont{white}{\color{white}}
%\DeclareCaptionFormat{listing}{\colorbox{gray}{\parbox{\textwidth}{#1#2#3}}}
%\captionsetup[lstlisting]{format=listing,labelfont=white,textfont=white}
%\usepackage{graphicx}
%\usepackage{amsmath, amssymb, amsthm}
%\usepackage[all,cmtip]{xy}
%\pagestyle{fancy}
\input{/home/dmitry/Work/Research/thesis/FINALE/settings.tex}

\begin{document}

\title{Diagnosing angular (directional) spectra by Tikhonov regularization (ridge regression, tapered least squares) in numerical studies of wave scattering (wave-topography interaction)}
\maketitle

\section*{Abstract}
Realistic numerical simulations of wave propagation (both surface and internal) involve complex patterns of interference that conceals energy transfers. The elementary wave components can be deduced from angular spectrum. Here, the energy distribution is found by solving an inverse model based on plane wave dynamics. The solution method is a least square minimization augmented with penalizing term (Tikhonov regularization). This aids stability in presence of noise and provides better spectral resolution for under-sampled signal. Further, applicability of the method for energy estimates is tested against analytical solution of wave scattering by bottom topography. In the first problem a periodic shallow gravity wave attacks a circular island. The inverse model correctly identifies energy lobes and their amplitude, hence, allowing to examine scattering amplitude in more complex environments. In the second test, a low mode internal wave reflects from a continental slope represented as step topography. Estimated bulk reflection coefficient compares excellent with the analytical solution. But in case of three dimensional topography the method as well provides information on spatial variation of point-wise coefficient due to inhomogeneity of the prescribed incident field. The results lend confidence in robustness of the inverse technique, hence, suggesting its applicability to field of the real ocean.

\section{Introduction}
The oceanic waves interfere and create obscure character (change?) in currents and pressure. The staggered pattern makes difficult to answer the simplest questions about waves: from where are they coming and how strong they are. Hence, there will always be an uncertainty in estimate of energy transfers and budgets. In an example of a standing wave, no energy transfer is taken place, yet two waves are present but oppositely traveling. The problem of diagnosing elementary components inevitably emerges in studies of wave-topography interaction. Energy is transfered into scattered wave field that undoubtedly has large number of components. This might produce spatially (horizontally) tight beams of energy transport. Such phenomena was observed in numerical simulations of tsunami waves (\cite{tang2012direct}) and internal waves of tidal frequency (internal tides) (\cite{simmons2004internal}, \cite{arbic2010concurrent}(?)) (and in observations (\cite{zhao2016global})). To present a method to deduce dynamics of a complex wave field such as aforementioned ones is a chief reason for this note (chapter, paper). And it is application to problems of tsunami wave scattering by submarine island and reflection of low-mode internal tide.\\

A straightforward method is to obtain spatial frequencies of elementary waves (\cite{barber1963directional}) by means of two dimensional Fourier transform. It further can be mapped onto a circle leading to angular spectra. In the field studies this approach was extensively used in relation to surface wave phenomena, e.g. propagation (\cite{munk1963directional}, reflection (\cite{dickson1995wave}, \cite{thomson2005reflection}, island diffraction (\cite{pawka1983island}). The spectral description was also applied to tsunami trapping (\cite{romano2013wavenumber}) and internal tide propagation (\cite{hendry1977observations}, \cite{lozovatsky2003spatial}). Estimation of spectra is a problem that one encounters and here is dealt by inverse modeling (\cite{long1986inverse}). As discrete Fourier transform leads to ambiguity in elementary wave direction, additional inferences are necessary. Even though application of the Hilbert transform can resolve the issue (\cite{mercier2008reflection}) \footnote{(Will it work with many components? In presence of corrugated surfaces?)}, it does not utilize all wave field characteristics available. Logically, constrains of wave dynamics are expected to lessen Gibbs phenomena, increase directional resolution and provide a way to super-resolution (\cite{kay1981spectrum}, \cite{sacchi1998interpolation}) so essential (viable, relevant, crucial) for resolving short-crested waves.\\
%the constrains of wave dynamics are expected to lessen

The inverse approach has to comprise dynamical statements of the wave field. So that angular spectrum will be related to (translated into) observed pressure and currents simultaneously. But such postulated problem leads to overdetermined and ill-posed system of equations, (least square) solution is unstable in relation to small changes or errors in data. For example, errors in numerical simulations might arise due to numerical and physical friction (reference?) leading to nonuniform wave amplitudes over sampling window of the numerical ocean. Hence, regularization is generally necessary to stabilize and improve inverse estimate (\cite{munk2009ocean}, \cite{snieder1999inverse}). In studies of wind wave spectra this is a well known approach when penalizing terms are added to cost functions. These terms can comprise constrains on spectrum shape and smoothness (\cite{long1979variational}, \cite{herbers1990estimation}), spectrum sparseness (\cite{hashimoto1989directional}, \cite{sacchi1996estimation}\footnote{this is geophysics reference}, or statistical properties (\cite{benoit1997analysing}). Unfortunately, the mentioned methods cannot be utilized in studies of wave-topography interaction. The scattered field is ``phase-locked" with incident one and thence, phase must be retained in the model equations (e.g. \cite{thomson2005reflection}) leading to deterministic wave decomposition. This is in contrary to wind waves of open sea that are mainly stochastic.\\

The simplest regularization technique, nevertheless powerful one seeks an inverse estimate of minimum variance. Initially developed by \cite{tikhonov2013numerical} for solution of Fredholm integral equations, Tikhonov regularization has a wide application starting from theoretical work in acoustics (\cite{colton1996simple}) and holography (\cite{williams2001regularization}) to observational oceanography (\cite{munk2009ocean}) and geophysics (\cite{snieder1999inverse}) where it is named ''tapered" or ``damped" least squares. But it has never been used \footnote{Dushaw used similar inverse} to oceanic wave problems. In the first section the inverse model will be formulated in order to find angular spectrum of waves. The developed approach than (Section 3) is illustrated on simple synthetic experiments to show its performance characteristics. And in later sections (4 and 5) is applied to numerical experiments of wave scattering where comparison with the well-known analytical solution is taken place. In section 6 concluding remarks are made and drawbacks are outlined.\\
\footnote{Should I add plane-wave fit? ``subtraction" method of Jody (and Zygmunt)?}
~\\

%The simplest way to identify the scattered components is to numerically remove bottom disturbance (\cite{kowalik2008kuril}, \cite{klymak2016reflection}). Than set new experiment to obtain differences which will provide insights into distribution of energy. This method can provide satisfactory results only if ``subtraction" of bathymetry is possible, that is there is no surrounding complexity. And if one is not interested in the energy transfers occurring over the topography which can be a case for a study on trapped wave motions. As well in complex numerical simulations when mesoscale conditions is simulated in the same time as wave phenomena, clearly removement of bathymetry will lead to different background physical conditions such that interaction between waves and background conditions will be erroneous.\\
%In classical observational wave () literature premises of linear wave field are used to obtain directional spectra, $|A(\sigma, \vec{k})|$. Fourier transform can be carried out in order to obtain frequency-wavenumber spectra. This defines distribution of wave energy in frequency-wavenumber space differential intervals. To obtain this different methods might be employed (). Here the trivial approach is considered based on plane-wave fit methods (). The basic approach to find spectral representation is to use underlying wave dynamics and use plane wave fit. This is a traditional approach in analysis of satellite altimetry observations (\cite{ray2001estimates}, \cite{zhao2016global}). In application of regular plane wave fit though there is a drawback that only finite number of components can be resolved. This problem arises as least squares method can infinitely increase amplitude in order to decrease misfit. This will lead to unphysical results. One of the ideas to solve this problem is setting additional restriction on amplitude. So this is why obtaining non-negative wavenumber spectra will bring better performance.\\
%In order to show feasibility of directional wave spectra in numerical experiments we proceed as follows. In section it is given the basic definition and description of numerical procedure. As well it is studied its characteristics on a synthetic data. With outlined procedure comparison of wavenumber spectra to an analytical solution of SGW diffraction by seamount is carried out (Section 3a) and comparison with reflection of internal tide from step-like continental slope is done with comparison with an appropriate analytical solutions (Section 3b). In section 4 concluding remarks are made.\\
%~\\

\section{Regularization technique of obtaining directional spectrum}
\subsection{Formulation of method}
Let pressure or sea level in complicated seas to be described by angular spectrum\footnote{This is not spectrum per se, is term usage still possible?},
\begin{equation}
\label{C1:eq.spectrum}
p(\vec{r}, t) = \int_0^{2\pi}  d\theta_k S(\theta_k) e^{i \vec{k}(\theta_k) \cdot \vec{r} + \phi(\theta_k) - i \omega t}
\end{equation}
Here each elementary (monochromatic) sine wave of wavenumber $k$ travels in direction $\theta$ with energy $S(\theta)^2 d\theta$ and temporal (spatial) lag of $\phi(\theta)$. Through out the note temporal dependence is omitted and complex notation is inferred. It is sought to estimate the angular spectrum $S(\theta)$ and phase distribution $\phi(\theta)$ from a finite set of observations. At first, this problem is reformulated in terms of polar\footnote{might not be correct term usage} Fourier coefficients (\cite{benoit1997analysing}, \cite{rafaely2004plane}) by Jacobi-Anger expansion \footnote{(should I cite anything?)},
\begin{equation}
p(r, \theta) = e^{i \vec{k}(\theta) \cdot \vec{r}} = \sum_{m = -\infty}^{m = \infty} i^{m} J_{m}(k r) e^{im(\theta - \theta_k)}
\end{equation}
shows that a field at point $(r, \theta)$ produced by plane wave can be expanded in series of Bessel functions and circular functions. Than its substitution into (\ref{C1:eq.spectrum}) and reorganization lead to
\begin{equation}
\label{C1:eq.series}
p(r, \theta) = \sum_{m=-\infty}^{m=\infty} \big[ \int_0^{2\pi}  d\theta_k S(\theta_k) e^{i\phi(\theta_k)} e^{-im\theta_k} \big] i^m J_m(kr) e^{im\theta}
\end{equation}
Term in brackets (square brackets) represent convolution integrals defining Fourier coefficients of order $m,~A_m - i B_m$. Thence, series (\ref{C1:eq.series}) state a model equation to find the unknown coefficients from the known, measured pressure field. Its spatial distribution can be sampled at a set of points $(r_i, \theta_i)$ and infinite series are truncated at order $N$ both leading to matrix formulation,
\begin{equation}
\label{C1:p.eq}
p_i = \sum_{m = -N}^{m = N} J_m(k r_i) e^{im(\theta + \pi/2)} (A_m  - i B_m)
\end{equation}
Real and imaginary parts constitute two separate problems allowing deterministic definition, so that angular spectra and phase distribution are found
\begin{align*}
S(\theta_k) e^{i \phi(\theta_k)} = \frac{1}{\pi} A_0 + \frac{2}{\pi} \big[ \sum_{m = 1}^{m = N} A_m \cos m\theta_k + i B_m \sin m\theta_k \big]
\end{align*}
The same steps are repeated but with current velocities instead of pressure, but starting from (\ref{C1:eq.spectrum}) transfer functions are inserted invoking plane wave dynamics,
\begin{align}
\begin{Bmatrix}
u(\vec{r}, t) \\ v(\vec{r}, t)
\end{Bmatrix}
=\int_0^{2\pi} d \theta_k \frac{k}{\rho_0 (\omega^2 - f^2)} 
\begin{Bmatrix}
\omega k \cos \theta_k + i f \sin \theta_k \\ \omega k \sin \theta_k - i f \cos \theta_k
\end{Bmatrix}
S(\theta_k) e^{i \vec{k}(\theta_k) \cdot \vec{r} + \phi(\theta_k)}
\end{align}
Than dependence of currents on wave bearing causes splitting of Fourier coefficients and asymmetry via Coriolis effect,
\begin{align}
\label{uv.eq}
\begin{Bmatrix}
u_i \\ v_i
\end{Bmatrix}
= \frac{1}{2} \sum_{m = -N}^{m = N} J_{m} (kr_i) e^{im(\theta + \pi/2)}
\begin{Bmatrix}
(\omega - f) A_{m + 1} + (\omega + f) A_{m - 1} - i [(\omega - f) B_{m + 1} + (\omega + f) B_{m - 1}] \\ 
(\omega - f) B_{m + 1} - (\omega + f) B_{m - 1} + i [ (\omega - f) A_{m + 1} - (\omega + f) A_{m - 1}]
\end{Bmatrix}
\end{align}
This is thought as a directional spreading or that velocity field is shifted towards red spectra - a well known phenomena for spatial Fourier transform (e.g. rhines)
Describe in words an inverse problem\\
Bessel functions decay... super fast\\
From where ill-posedness rises?\\
The so-chosen basis functions are not orthogonal, this leads to ill-conditioned matrix K, the solution is unstable for small changes in the data. Some sort of regularization is necessary. Here we choose the simplest of most is Tikhonov regularization. The minimization problem to solve is (or to use cost function)
\begin{equation}
\label{Tikh_prob}
\text{min}\{||K x - y||^2_2 + \alpha ||x||^2_2\}
\end{equation}
Here the first term is part of regular least-squares solution and represents minimization of residual. The second term minimizes variance. One tries to find a balanced solution that has minimum residual and smallest norm trade-off parameter.
But success and resultant fit depends on a regularization parameter. The parameter acts as a high pass filter over eigenvectors of $K$ (\cite{williams2001regularization})\\
In case of rapidly varying bathymetry a local cutoff to drop stations is based on WKB-alike condition, i.e. variations in wavenumber are much smaller compare to its value.\\
The formed matrix equations is solved by means of linear programming methods \cite{haskell1981algorithm}.\\

\subsection{Example from synthetic experiment}


\subsection{Numerical implementation}
For purposes of numerical analysis we will use a circular antenna (Figure 2a). Its directional resolving power (\cite{barber1963directional}) is given by Fig. 1b,
\begin{equation}
G(k,l) = 1 + \sum_{i,~j} 2\cos(2\pi \vec{k} \vec{\Delta_{ij}})
\end{equation}
where $i,j$ goes every distance between array points.\\
\begin{figure}
\mfig[0.5]{dir_resolution.png}
\caption{Array organization and its direction resolution (middle) and array response function (right).}
\end{figure}
To test the developed method it is carried out experiments with synthetic data. The considered data are representative of the problems in following up comparisons with analytical solution. In the first problem 4 waves represent scattering by seamount when one incident wave generates two side lobes and one reflected. Results are given by . It shows that the presented method .....\\
In the second problem incident wave is interfering with reflected. Here phase is underoging random variations.\\
WHY I AM DOING THE SYNTHETIC EXPERIMENTS? TO ESTIMATE ERRORS.\\
\small{(((($b_{ij}(\theta)$ - kernel matrix that transforms directional decomposition into physical observations)))).}\\

\subsection{Analytical example, resolution}
To investigate properties of the wave spectra let study the analytical expressions for simplified applications.\\
Consider an array consisting of two points with coordinates $\vec{x}$ and $\vec{x} + \delta \vec{x}$ and two plane waves $\psi_1 = A_1 e^{i (\vec{k_1} \cdot \vec{x})}$ and $\psi_2 = A_2 e^{i (\vec{k_2} \cdot \vec{x} + \phi)}$. They interfere ($\psi = \psi_1 + \psi_2$) generating the following expressions for cross-spectrum:
\begin{subequations}
\begin{align}
C_{11} = \psi (\vec{x}) \cj{\psi(\vec{x})} = A_1^2 + A_2^2 + 2A_1A_2 \cos(\Delta \vec{k} \vec{x} - \phi) \label{C1:examp.a}\\
C_{12} = \psi (\vec{x}) \cj{\psi(\vec{x} + \delta \vec{x})} = 
 A_1^2 e^{-i \vec{k}_1 \delta \vec{x}} + A_2^2 e^{-i \vec{k}_2 \delta \vec{x}} + 2 A_1 A_2 \cos(\Delta \vec{k} \vec{x} - \phi - \frac{\Delta \vec{k}}{2} \delta \vec{x}) e^{-i \frac{\vec{k}_s}{2} \delta \vec{x}} \label{C1:examp.b}\\
C_{22} =  \psi (\vec{x} + \delta \vec{x}) \cj{\psi(\vec{x} + \delta \vec{x})} = 
A_1^2 + A_2^2 + 2A_1A_2 \cos(\Delta \vec{k} (\vec{x} + \delta \vec{x}) - \phi) \label{C1:examp.c}
\end{align}
\end{subequations}
with $\Delta \vec{k} = \vec{k}_1 - \vec{k}_2 = 2|\vec{k}| \sin \frac{\delta \alpha}{2}$. These expressions consist of parts that represent amplitude of the waves and variable signal originating from difference in waves propagation or due to antenna resolution. The proposed method tries to identify magnitudes from $C_{11}$ and $C_{22}$ and direction from $C_{12}$ since autocovariance is subject to fit $\sum_i S_i$ and crosscovariance - to $\sum_i S_i e^{-i k_i \delta \vec{x}}$.\\
%\small{(((((($C_{11} = A_1^2 + A_2^2 + 2A_1A_2 \cos(\Delta \vec{k} \vec{x} - \phi)$ is subject to fit for $\sum_i S_i + i*0$.\\
%$C_{12} = A_1^2 e^{-i \vec{k}_1 \delta \vec{x}} + A_2^2 e^{-i \vec{k}_2 \delta \vec{x}} + A_1 A_2 \big( e^{i(\Delta \vec{k} \vec{x} - \vec{k}_2 \delta \vec{x} - \phi)} + e^{i(-\Delta \vec{k} \vec{x} - \vec{k}_1 \delta \vec{x} + \phi)}\big)$ is subject to fit for $\sum_l S_l \cos(k_l \delta \vec{x}) - i*\sum_l S_l \sin(k_l \delta \vec{x})$)))))}\\
The errors in spectra will arise if variable parts, e.g. $2A_1A_2 \cos(\Delta \vec{k} \vec{x} - \phi)$,  of cross-spectrum will be large leading to aliasing of energy. In case of aliasing the numerical method cannot distinguish two waves either in their amplitude or direction. The best performance to obtain energy will occur when, \eqref{C1:examp.c}, $\cos (\Delta \vec{k} \delta \vec{x}) = 0,~\Delta \vec{k} \delta \vec{x} = \frac{\pi}{2} (2n + 1)$. Oppositely, from \eqref{C1:examp.b} to destroy energy leakage into beat wavenumber $\vec{k}_s$, $\cos (\Delta \frac{1}{2}\vec{k} \delta \vec{x}) = 0,~\Delta \vec{k} \delta \vec{x} = \pi(2n + 1)$. These two conditions lead to uncertainty, that this antenna cannot perform well both for magnitude and direction. 

\subsection{Synthetic experiments}
To test ability of the proposed method and antenna array I carry out synthetic experiments. The aim is to understand how the method can resolve multiple-wave interference. In the scattering events usually it can be found incident (here taken to be 0 degrees), side energy lobe (45 degrees), reflected wave (210 degrees). These three waves are kept constant with amplitude of 1, 0.5, 0.5. Than additionally it is prescribed three wave components with random amplitude, phase and direction. Amplitude is in range from 0 to 0.1. These random wave components represent a noise suggestive of some numerical deficiencies. The goal is to understand what is the uncertainty in defining the three primary components.\\


\section{Application}
\subsection{Tsunami wave diffraction by a round island}
Now the method is tested against known analytical problems. The surface gravity wave interaction with submarine seamount. THe solution is known for many years (\cite{longuet1967trapping}).\\
In the numerical experiment it is used the simplified numerical code developed by (\cite{kowalik2005numerical}). All nonlinear terms were neglected, while bottom friction is controlled by constant coefficient of $3.3 \cdot 10^{-3}$. The integration is carried out over grid with resolution of $2~km$ by $2~km$. The continuous harmonic wave train is sent towards the shoal of depth $500~m$ in sea of depth $5300~m$ from the east side. After the numerical experiment achieved equilibrium the sea level and horizontal currents are sampled. Fourier transform is done and than the presented method here is used to build directional distribution in the wave field. Example map of energy flux is presented on Figure 3.\\
\begin{figure}
\mfig[0.5]{fluxes_circ_sm.png}
\caption{The simulated energy flux maps for SGW diffraction over the shoal.}
\end{figure}
The directional decomposition shows 
\begin{figure}
\mfig[0.5]{spectra_windrose_circ_sm.png}
\caption{The simulated energy flux maps for SGW diffraction over the shoal.}
\end{figure}
The analytical solution is shown on Figure 4.
\begin{figure}
\mfig[0.5]{scatt_LH_periods.png}
\caption{analytical solution}
\end{figure}

\subsection{Internal tide reflection from step}
For internal tide reflection problem it is taken MITgcm numerical model initialized with constant stratification. The continental slope is represented as depth discontinuity arising from flat bottom at 2000 m to variable depth. The numerical results after the equailibrium was achieved are sample, harmonically fitted to $M_2$ frequency. Than eigenmode decomposition is done. Example of the obtained results is shown by Figure ...\\
\begin{figure}
\mfig[0.5]{just_interf.png}
\caption{Interference due to oblique reflection.}
\end{figure}
The solution is compared against \cite{chapman1981scattering} and is shown by
\begin{figure}
\mfig[0.5]{refl_coef_smp.png}
\caption{Reflection coefficient}
\end{figure}

\section{Discussion and Conclusions}
\begin{itemize}
\item Energy preservation
\item No phase reconstruction
\item Discrete since no a priori model is used
\item Useful tool for numerical experiment analysis
\end{itemize}

\bibliographystyle{apacite}
\bibliography{/home/dmitry/Bibtex_lib/my_first_lib}

\section{Problems}
\begin{itemize}
\item I am confusing spectra and amplitude, as well cross-spectrum and covariance
\end{itemize}
\section{TO DO LIST}
\begin{itemize}
\item Right now my questions are weak. Primarily, it comes from my explanation, why do you need something new? My major answer: a) regular plane wave fit fails for multiple wave components; b) in complex models - it is not that easy to subscribe bathymetry. \textbf{That should be my arguments!} Explore them in more detail. Ref: Zhao method, Mercier, Jody's subtraction.

\item Monte Carlo and estimate of errors - first thing to do

\item Solution for wave scattering by an elliptic seamount

\item What is physical meaning of Sf? Ref: Wagner, IT scattering

\item Set Chapman experiments

\item Set tsunami experiments with Koko Guoyt.
\end{itemize}
\section{pieces}
Here the cross-spectral components of observations at points $i$ and $j$; $S(\theta)$ - directional wave spectra in direction $\theta$ and . Here for specifying this relation the regular plane wave representation with corresponding polarization relations is adopted. For example, cross-spectra between velocity at point $i$ and pressure at point $j$ (somewhat related to "energy flux") will be expressed as\\

with $\vec{k}(\theta) = k (\cos(\theta), \sin (\theta))$ and $\vec{x}_i,~\vec{x}_j$ will be radius vectors to points $i$ and $j$. Numerically, the integral representation is disretized and linear model to solve at each computation grid cell\\

\end{document}
