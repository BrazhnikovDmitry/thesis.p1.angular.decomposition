%\documentclass[12pt]{article}
%\usepackage[margin=1 in, head=0.9 in]{geometry}
%\usepackage{fancyhdr}
%\usepackage{listings}
%\usepackage{caption}
%\usepackage{color}
%\usepackage{xcolor}
%\usepackage{caption, apacite}
%\DeclareCaptionFont{white}{\color{white}}
%\DeclareCaptionFormat{listing}{\colorbox{gray}{\parbox{\textwidth}{#1#2#3}}}
%\captionsetup[lstlisting]{format=listing,labelfont=white,textfont=white}
%\usepackage{graphicx}
%\usepackage{amsmath, amssymb, amsthm}
%\usepackage[all,cmtip]{xy}
%\pagestyle{fancy}
\input{/home/dmitry/Work/Research/thesis/FINALE/settings.tex}

\begin{document}

\title{Application of angular spectra in analysis of gravity wave diffraction, scattering, reflection in the numerical models}
\maketitle

\section{Abstract}
Wave transports of mechanical energy are important feature of the World ocean shaping physical processes on different scales. Under conditions of multiple source waves the energy transports can be masked by intrinsic wave phenomena of superposition and interference. \textit{Hence, it is crucial to have a method for decomposition of an observed wave pattern into parts that correspond for particular sources. Here it is proposed an extension of directional spectral decomposition for analysis of surface gravity waves (tsunami) and internal waves of tidal frequency (internal tides).} The method is based on plane wave fitting into coherence between superposed wave associated currents and pressure. By fitting analytical model into quad- and co-spectral components it is obtained directional distribution of energy. This representation is further used in order to study tsunami wave diffraction by an island and reflection of internal tides from a step bathymetry representative of continental shelf with comparison to the known analytical solutions. The method is shown to be useful in understanding of energy transports associated with particular waves of interest and can found wide spread application to observations and satellite altimetry observations.

\section{Introduction}
The waves interfere and create obscure patterns of energy transfer. The standing wave is a classical example in which no energy transfer is occurring.\\
In analysis of the numerical experiments it is important to decompose the complex wave field in order to understand sources, outline energy budgets.\\
Here a simple method of decomposition is proposed based on the work of \cite{long1979variational}
How to deal with it? Different methods were proposed. Here we intend to present a method to describe the field in terms of angular spectra.

\section{Fitting technique}
\subsection{Method of obtaining directional spectra}
Just description using normal language. Also give an analytical example. Define coherent/incoherent parts.\\
The fitting technique is given by
\begin{equation}
C_{ij} = \int b_{ij}(\theta) S(\theta) d \theta
\end{equation}
Here the cross-spectral components of observations at points $i$ and $j$; $S(\theta)$ - directional wave spectra in direction $\theta$ and $b_{ij}(\theta)$ - kernel matrix that 
transforms directional decomposition into physical observations. Here for specifying this relation the regular plane wave representation with corresponding polarization relations is adopted. For example, cross-spectra between velocity at point $i$ and pressure at point $j$ (somewhat related to "energy flux") will be expressed as
\begin{equation}
u_i^{\star} \cdot p_j = \int \frac{\omega k \cos \theta + i fk \sin \theta}{\rho(\omega^2 - f^2)} e^{-i \vec{k}(\theta) \cdot \vec{x}_i} e^{i \vec{k}(\theta) \cdot \vec{x}_j}  S(\theta) d \theta
\end{equation}
with $\vec{k}(\theta) = k (\cos(\theta), \sin (\theta))$ and $\vec{x}_i,~\vec{x}_j$ will be radius vectors to points $i$ and $j$. Numerically, the integral representation is disretized and linear model to solve at each computation grid cell
\begin{align*}
&&C_{ij} = \sum_k \textbf{b}_{ij}^k S_k \Delta \theta\\
&&S_k \geq 0
\end{align*}
This system of equations is formed for each pair of the points. These observational stations arranged as circular pattern antennas. The radius is defined as local wavelength found from the corresponding dispersion relation.\\
In case of rapidly varying bathymetry a local cutoff to drop stations is based on WKB-alike condition, i.e. variations in wavenumber are much smaller compare to its value.

\subsection{Analytical example}

\subsection{Synthetic experiments (Monte Carlo simulations and Errors)}
Consider really simple case: two wave interaction though with different amplitudes, change their direction of propagation and phase. Here use both grid and antenna, but take simple as possible as parameter. Goal is to obtain $\delta x \cdot \delta E \leq value$. Then only free parameter for those antenna which will be distance between points, so shuffle them around.

\section{Tsunami wave diffraction by a round island}
\subsection{Analytical solution}
Following previous section figure out errors, what affects them? Maybe Monte Carlo simulation with some random spectral characteristic. Point here is to understand what can be expected in complex wave interference cases\\
Create some metric based on WKB and antenna properties to estimate the error.
\subsection{Application}

\section{Internal tide reflection from step}
\subsection{Analytical solution}
\subsection{Application}

\section{Discussion and Conclusions}
How well the technique can do in presence of non-plane wave fields, i.e. When there is rapid change in bathymetry. It is some sort of conclusion part\\
When there is rapid change in bathymetry and there is a long wave, the antenna loses many points. Thus, adjust antenna radii in accordance with local bathymetry.

\bibliographystyle{apacite}
\bibliography{/home/dmitry/Bibtex_lib/}

\section{TO DO LIST}
\begin{itemize}
\item Right now my questions are weak. Primarily, it comes from my explanation, why do you need something new? My major answer: a) regular plane wave fit fails for multiple wave components; b) in complex models - it is not that easy to subscribe bathymetry. \textbf{That should be my arguments!} Explore them in more detail. Ref: Zhao method, Mercier, Jody's subtraction.

\item Monte Carlo and estimate of errors - first thing to do

\item Solution for wave scattering by an elliptic seamount

\item What is physical meaning of Sf? Ref: Wagner, IT scattering

\item Set Chapman experiments

\item Set tsunami experiments with Koko Guoyt.
\end{itemize}
\end{document}