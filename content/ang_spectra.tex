%\documentclass[12pt]{article}
%\usepackage[margin=1 in, head=0.9 in]{geometry}
%\usepackage{fancyhdr}
%\usepackage{listings}
%\usepackage{caption}
%\usepackage{color}
%\usepackage{xcolor}
%\usepackage{caption, apacite}
%\DeclareCaptionFont{white}{\color{white}}
%\DeclareCaptionFormat{listing}{\colorbox{gray}{\parbox{\textwidth}{#1#2#3}}}
%\captionsetup[lstlisting]{format=listing,labelfont=white,textfont=white}
%\usepackage{graphicx}
%\usepackage{amsmath, amssymb, amsthm}
%\usepackage[all,cmtip]{xy}
%\pagestyle{fancy}
\input{/home/dmitry/Work/Research/thesis/FINALE/settings.tex}

\begin{document}

\title{Diagnosing wave energy transfers in numerical studies of scattering by directional spectra (Be careful with energy transfers)}
\maketitle

\section*{Abstract}
WHAT DO YOU WANNA PRESENT? WHICH ONE OF THESE?
\begin{itemize}
\item Method
\item Usage of method
\item Scattering
\end{itemize}
Right now DDEC is used only for calculation of reflection coefficient and really for nothing in tsunami, but I wanted to use it for identification of scattering energy lobes and maybe filtering them.\\
Ok, this is ``an approach for analysis wave scattering by ddec".
Another note: I don't use DDEC for tsunami event, or maybe somehow identify energy lobes? Also you do inverse model.\\

Realistic numerical simulations of wave propagation usually involve complex patterns of interference that conceals energy transfers. The elementary wave components can be deduced from directional spectra. Here, the energy distribution is found by solving an inverse model. The method is based on plane wave fitting into \textbf{coherence} between wave field associated currents and pressure. By fitting analytical model into quad- and co-spectral components the obtained directional distribution of energy provides a way to quantify energy transports after scattering event. This representation is further used in order to study tsunami wave diffraction by an island and reflection of internal tide from a step bathymetry representative of continental shelf. Comparison with known analytical solutions emphasizes ability of the proposed method in delineating energy budgets for complex wave fields.\\

 The method is shown to be useful in understanding of energy transports associated with particular waves of interest\textbf{ and can found wide spread application to observations and satellite altimetry observations}.\\
Example, abstract\\
We present here a new InSAR persistent scatterer (PS) method for analyzing episodic crustal deformation in non-urban environments, with application to volcanic settings. Our method for identifying PS pixels in a series of interferograms is based primarily on phase characteristics and finds low-amplitude pixels with phase stability that are not identified by the existing amplitude-based algorithm. Our method also uses the spatial correlation of the phases rather than a well-defined phase history so that we can observe temporally-variable processes, e.g., volcanic deformation. The algorithm involves removing the residual topographic component of flattened interferogram phase for each PS, then unwrapping the PS phases both spatially and temporally. Our method finds scatterers with stable phase characteristics independent of amplitudes associated with man-made objects, and is applicable to areas where conventional InSAR fails due to complete decorrelation of the majority of scatterers, yet a few stable scatterers are present.\\


\section{Introduction}
The oceanic waves interfere and create obscure patterns in wave associated currents and pressure. This further masks energy propagation and create uncertainty in estimation of energy budgets. In a classical example of a standing wave no energy transfer is taken place, yet two wave components can be distinguished. This problem of diagnosing elementary components  inevitably emerges in studies of wave-topography interaction. Energy is transfered into scattered wave field which than superposes with the primary component. To quantify this process is a primary reason for this note.\\
\textit{Geophysical examples}
The scattered wavefield has large number of wave components whose propagation might be quite different from the initial incidence. After scattering event it might be produced tight beams of energy transport or for energy flux - gyres. Such phenomena was found in numerical simulations  of tsunami waves (\cite{tang2012direct}) and observed for internal waves of tidal frequency (internal tides) (\cite{zhao2016global}). This work intends to analyze spatial distribution of energy. (Here outline two problems of geophysical interest).\\
The simplest way to identify the scattered components is to numerically remove bottom disturbance (\cite{kowalik2008kuril}, \cite{klymak2016reflection}). Than set new experiment to obtain differences which will provide insights into distribution of energy . This method can provide satisfactory results only if ``subtraction" of bathymetry is possible, that is there is no surrounding complexity. And if one is not interested in the energy transfers occurring over the topography which can be a case for a study on trapped wave motions. As well in complex numerical simulations when mesoscale conditions is simulated in the same time as wave phenomena, clearly removement of bathymetry will lead to different background physical conditions such that interaction between waves and background conditions will be erroneous.\\
k-f analysis, For problems when direction of scattered components is opposite to each other one can use techniques based on spatial filtering such as complex demodulation (Hilbert transform)\cite{mercier2008reflection}(???). In multiple wavefield such as one arousing from wave interaction with seamounts this will not give a satisfactory result.\\
In classical observational wave () literature premises of linear wave field are used to obtain directional spectra, $|A(\sigma, \vec{k})|$. Fourier transform can be carried out in order to obtain frequency-wavenumber spectra. This defines distribution of wave energy in frequency-wavenumber space differential intervals. To obtain this different methods might be employed (). Here the trivial approach is considered based on plane-wave fit methods (). The basic approach to find spectral representation is to use underlying wave dynamics and use plane wave fit. This is a traditional approach in analysis of satellite altimetry observations (\cite{ray2001estimates}, \cite{zhao2016global}). In application of regular plane wave fit though there is a drawback that only finite number of components can be resolved. This problem arises as least squares method can infinitely increase amplitude in order to decrease misfit. This will lead to unphysical results. One of the ideas to solve this problem is setting additional restriction on amplitude. So this is why obtaining non-negative wavenumber spectra will bring better performance.\\
The application of wavenumber spectra was done in field observations for surface gravity wave reflection (\cite{dickson1995wave}) and interaction with underlying bottom topography (\cite{thomson2005reflection}). The same principles are used here in application to simplified numerical experiments of surface wave scattering by underwater seamount and internal tide reflection from continental shelf.\\
In order to show feasibility of directional wave spectra in numerical experiments we proceed as follows. In section it is given the basic definition and description of numerical procedure. As well it is studied its characteristics on a synthetic data. With outlined procedure comparison of wavenumber spectra to an analytical solution of SGW diffraction by seamount is carried out (Section 3a) and comparison with reflection of internal tide from step-like continental slope is done with comparison with an appropriate analytical solutions (Section 3b). In section 4 concluding remarks are made.

\section{Fitting technique}
\subsection{Method of obtaining directional spectra}
Here for application the unifrequency component is described. So that the wave field is represented by its wavenumber spectra described by directional propagation of respective linear wave component (\cite{long1986inverse}),
\begin{equation}
p(\vec{x}, t) = \int_{0}^{2\pi} b_{p}(\theta) A(\theta) e^{i \vec{k}(\theta) \cdot \vec{x} - \omega t} d \theta
\end{equation}
where $p$ are pressure, $A(\theta)$ - differential amplitude of wave traveling at angle $\theta$. $b_{u,v,p}$ is the corresponding transfer function relating spectral representation to physical quantities.\\
Than cross-spectrum components between observations distanced by $\vec{r}$ is easily found,
\begin{equation}
p_i(\vec{x}) p_j(\vec{x} + \vec{r}) = \int_{0}^{2\pi} b_{i}(\theta) b_{j}(\theta) (A(\theta))^2 e^{i \vec{k}(\theta) \cdot \vec{r}} d \theta \label{C1:mmodel}
\end{equation}
By obtaining the cross-spectrum components at $N$ numerical-grid points between parameters of simulated wave field the directional distribution of amplitude, hence, can be obtained.\\
The transfer functions for pressure and currents follow dynamical considerations of the waves. Such that for pressure $b_{p}$ is $1$ and for currents corresponding polarization relations can be used. For example, if the cross-spectrum between current and pressure is found, the model \eqref{C1:mmodel} will have the form of
\begin{equation}
u_i^{\star} \cdot p_j = \int \frac{\omega k \cos \theta + i fk \sin \theta}{\rho(\omega^2 - f^2)} e^{-i \vec{k}(\theta) \cdot \vec{x}_i} e^{i \vec{k}(\theta) \cdot \vec{x}_j}  S(\theta) d \theta
\end{equation}
Since in the numerical models, the physical quantities are not random the simple least square with additional condition can be used. Forming matrix equation, the model to minimize will be,
\begin{subequations}
\begin{align}
C_{ij} = \sum_k \textbf{b}_{ij}^k S_k \Delta \theta \label{C1:ls.a} \\
S_k \geq 0 \label{C1:ls.b}
%\end{split}
\end{align}
\end{subequations}
where $C_{ij}$ represents cross-spectrum between pressure and currents at points $i$ and $j$ in ``observational" array, $b_{ij}(\theta)$ corresponding transfer function and $S(\theta_k)$ - directional spectral density function discretized. The circle is discretized at $\theta_k$. The posed problem is classical least square problem for \eqref{C1:ls.a}, but additionally the solution is restricted by non-negativity of spectra. This provides constraint on least squares and prevents spurious amplitudes.\\
This system of equations is formed for each pair of the points. These observational stations arranged as circular pattern antennas. The radius is defined as local wavelength found from the corresponding dispersion relation.\\
In case of rapidly varying bathymetry a local cutoff to drop stations is based on WKB-alike condition, i.e. variations in wavenumber are much smaller compare to its value.\\
The formed matrix equations is solved by means of linear programming methods \cite{haskell1981algorithm}.\\
\subsection{Numerical implementation}
For purposes of numerical analysis we will use a circular antenna (Figure 2a). Its directional resolving power (\cite{barber1963directional}) is given by Fig. 1b,
\begin{equation}
G(k,l) = 1 + \sum_{i,~j} 2\cos(2\pi \vec{k} \vec{\Delta_{ij}})
\end{equation}
where $i,j$ goes every distance between array points.\\
\begin{figure}
\mfig[0.5]{dir_resolution.png}
\caption{Array organization and its direction resolution (middle) and array response function (right).}
\end{figure}
To test the developed method it is carried out experiments with synthetic data. The considered data are representative of the problems in following up comparisons with analytical solution. In the first problem 4 waves represent scattering by seamount when one incident wave generates two side lobes and one reflected. Results are given by . It shows that the presented method .....\\
In the second problem incident wave is interfering with reflected. Here phase is underoging random variations.\\
WHY I AM DOING THE SYNTHETIC EXPERIMENTS? TO ESTIMATE ERRORS.\\
\small{(((($b_{ij}(\theta)$ - kernel matrix that transforms directional decomposition into physical observations)))).}\\

\subsection{Analytical example, resolution}
To investigate properties of the wave spectra let study the analytical expressions for simplified applications.\\
Consider an array consisting of two points with coordinates $\vec{x}$ and $\vec{x} + \delta \vec{x}$ and two plane waves $\psi_1 = A_1 e^{i (\vec{k_1} \cdot \vec{x})}$ and $\psi_2 = A_2 e^{i (\vec{k_2} \cdot \vec{x} + \phi)}$. They interfere ($\psi = \psi_1 + \psi_2$) generating the following expressions for cross-spectrum:
\begin{subequations}
\begin{align}
C_{11} = \psi (\vec{x}) \cj{\psi(\vec{x})} = A_1^2 + A_2^2 + 2A_1A_2 \cos(\Delta \vec{k} \vec{x} - \phi) \label{C1:examp.a}\\
C_{12} = \psi (\vec{x}) \cj{\psi(\vec{x} + \delta \vec{x})} = 
 A_1^2 e^{-i \vec{k}_1 \delta \vec{x}} + A_2^2 e^{-i \vec{k}_2 \delta \vec{x}} + 2 A_1 A_2 \cos(\Delta \vec{k} \vec{x} - \phi - \frac{\Delta \vec{k}}{2} \delta \vec{x}) e^{-i \frac{\vec{k}_s}{2} \delta \vec{x}} \label{C1:examp.b}\\
C_{22} =  \psi (\vec{x} + \delta \vec{x}) \cj{\psi(\vec{x} + \delta \vec{x})} = 
A_1^2 + A_2^2 + 2A_1A_2 \cos(\Delta \vec{k} (\vec{x} + \delta \vec{x}) - \phi) \label{C1:examp.c}
\end{align}
\end{subequations}
with $\Delta \vec{k} = \vec{k}_1 - \vec{k}_2 = 2|\vec{k}| \sin \frac{\delta \alpha}{2}$. These expressions consist of parts that represent amplitude of the waves and variable signal originating from difference in waves propagation or due to antenna resolution. The proposed method tries to identify magnitudes from $C_{11}$ and $C_{22}$ and direction from $C_{12}$ since autocovariance is subject to fit $\sum_i S_i$ and crosscovariance - to $\sum_i S_i e^{-i k_i \delta \vec{x}}$.\\
%\small{(((((($C_{11} = A_1^2 + A_2^2 + 2A_1A_2 \cos(\Delta \vec{k} \vec{x} - \phi)$ is subject to fit for $\sum_i S_i + i*0$.\\
%$C_{12} = A_1^2 e^{-i \vec{k}_1 \delta \vec{x}} + A_2^2 e^{-i \vec{k}_2 \delta \vec{x}} + A_1 A_2 \big( e^{i(\Delta \vec{k} \vec{x} - \vec{k}_2 \delta \vec{x} - \phi)} + e^{i(-\Delta \vec{k} \vec{x} - \vec{k}_1 \delta \vec{x} + \phi)}\big)$ is subject to fit for $\sum_l S_l \cos(k_l \delta \vec{x}) - i*\sum_l S_l \sin(k_l \delta \vec{x})$)))))}\\
The errors in spectra will arise if variable parts, e.g. $2A_1A_2 \cos(\Delta \vec{k} \vec{x} - \phi)$,  of cross-spectrum will be large leading to aliasing of energy. In case of aliasing the numerical method cannot distinguish two waves either in their amplitude or direction. The best performance to obtain energy will occur when, \eqref{C1:examp.c}, $\cos (\Delta \vec{k} \delta \vec{x}) = 0,~\Delta \vec{k} \delta \vec{x} = \frac{\pi}{2} (2n + 1)$. Oppositely, from \eqref{C1:examp.b} to destroy energy leakage into beat wavenumber $\vec{k}_s$, $\cos (\Delta \frac{1}{2}\vec{k} \delta \vec{x}) = 0,~\Delta \vec{k} \delta \vec{x} = \pi(2n + 1)$. These two conditions lead to uncertainty, that this antenna cannot perform well both for magnitude and direction. 

\subsection{Synthetic experiments}
To test ability of the proposed method and antenna array I carry out synthetic experiments. The aim is to understand how the method can resolve multiple-wave interference. In the scattering events usually it can be found incident (here taken to be 0 degrees), side energy lobe (45 degrees), reflected wave (210 degrees). These three waves are kept constant with amplitude of 1, 0.5, 0.5. Than additionally it is prescribed three wave components with random amplitude, phase and direction. Amplitude is in range from 0 to 0.1. These random wave components represent a noise suggestive of some numerical deficiencies. The goal is to understand what is the uncertainty in defining the three primary components.\\


\section{Application}
\subsection{Tsunami wave diffraction by a round island}
Now the method is tested against known analytical problems. The surface gravity wave interaction with submarine seamount. THe solution is known for many years (\cite{longuet1967trapping}).\\
In the numerical experiment it is used the simplified numerical code developed by (\cite{kowalik2005numerical}). All nonlinear terms were neglected, while bottom friction is controlled by constant coefficient of $3.3 \cdot 10^{-3}$. The integration is carried out over grid with resolution of $2~km$ by $2~km$. The continuous harmonic wave train is sent towards the shoal of depth $500~m$ in sea of depth $5300~m$ from the east side. After the numerical experiment achieved equilibrium the sea level and horizontal currents are sampled. Fourier transform is done and than the presented method here is used to build directional distribution in the wave field. Example map of energy flux is presented on Figure 3.\\
\begin{figure}
\mfig[0.5]{fluxes_circ_sm.png}
\caption{The simulated energy flux maps for SGW diffraction over the shoal.}
\end{figure}
The directional decomposition shows 
\begin{figure}
\mfig[0.5]{spectra_windrose_circ_sm.png}
\caption{The simulated energy flux maps for SGW diffraction over the shoal.}
\end{figure}
The analytical solution is shown on Figure 4.
\begin{figure}
\mfig[0.5]{scatt_LH_periods.png}
\caption{analytical solution}
\end{figure}

\subsection{Internal tide reflection from step}
For internal tide reflection problem it is taken MITgcm numerical model initialized with constant stratification. The continental slope is represented as depth discontinuity arising from flat bottom at 2000 m to variable depth. The numerical results after the equailibrium was achieved are sample, harmonically fitted to $M_2$ frequency. Than eigenmode decomposition is done. Example of the obtained results is shown by Figure ...\\
\begin{figure}
\mfig[0.5]{just_interf.png}
\caption{Interference due to oblique reflection.}
\end{figure}
The solution is compared against \cite{chapman1981scattering} and is shown by
\begin{figure}
\mfig[0.5]{refl_coef_smp.png}
\caption{Reflection coefficient}
\end{figure}

\section{Discussion and Conclusions}
\begin{itemize}
\item Energy preservation
\item No phase reconstruction
\item Discrete since no a priori model is used
\item Useful tool for numerical experiment analysis
\end{itemize}

\bibliographystyle{apacite}
\bibliography{/home/dmitry/Bibtex_lib/my_first_lib}

\section{Problems}
\begin{itemize}
\item I am confusing spectra and amplitude, as well cross-spectrum and covariance
\end{itemize}
\section{TO DO LIST}
\begin{itemize}
\item Right now my questions are weak. Primarily, it comes from my explanation, why do you need something new? My major answer: a) regular plane wave fit fails for multiple wave components; b) in complex models - it is not that easy to subscribe bathymetry. \textbf{That should be my arguments!} Explore them in more detail. Ref: Zhao method, Mercier, Jody's subtraction.

\item Monte Carlo and estimate of errors - first thing to do

\item Solution for wave scattering by an elliptic seamount

\item What is physical meaning of Sf? Ref: Wagner, IT scattering

\item Set Chapman experiments

\item Set tsunami experiments with Koko Guoyt.
\end{itemize}
\section{pieces}
Here the cross-spectral components of observations at points $i$ and $j$; $S(\theta)$ - directional wave spectra in direction $\theta$ and . Here for specifying this relation the regular plane wave representation with corresponding polarization relations is adopted. For example, cross-spectra between velocity at point $i$ and pressure at point $j$ (somewhat related to "energy flux") will be expressed as\\

with $\vec{k}(\theta) = k (\cos(\theta), \sin (\theta))$ and $\vec{x}_i,~\vec{x}_j$ will be radius vectors to points $i$ and $j$. Numerically, the integral representation is disretized and linear model to solve at each computation grid cell\\

\end{document}